\documentclass[conference]{doc}
\usepackage{blindtext, graphicx}
\ifCLASSINFOpdf
  % \usepackage[pdftex]{graphicx}
  % declare the path(s) where your graphic files are
  % \graphicspath{{../pdf/}{../jpeg/}}
  % and their extensions so you won't have to specify these with
  % every instance of \includegraphics
  % \DeclareGraphicsExtensions{.pdf,.jpeg,.png}
\else
  % or other class option (dvipsone, dvipdf, if not using dvips). graphicx
  % will default to the driver specified in the system graphics.cfg if no
  % driver is specified.
  % \usepackage[dvips]{graphicx}
  % declare the path(s) where your graphic files are
  % \graphicspath{{../eps/}}
  % and their extensions so you won't have to specify these with
  % every instance of \includegraphics
  % \DeclareGraphicsExtensions{.eps}
\fi

% correct bad hyphenation here
\hyphenation{op-tical net-works semi-conduc-tor}

\begin{document}
	\title{Extracting tourists movements' information from Twitter data sets}
	\author{
		\IEEEauthorblockN{Micka\"{e}l Nkunku}
		\IEEEauthorblockA{
			Enschede, Netherlands
		}
		\and
		\IEEEauthorblockN{Thibault Defeyter}
		\IEEEauthorblockA{
			Enschede, Netherlands
		}
		\and
		\IEEEauthorblockN{Jose Ailton Filho}
		\IEEEauthorblockA{
			Enschede, Netherlands
		}
		\and
		\IEEEauthorblockN{Fernando Augusto Machado}
		\IEEEauthorblockA{
			Enschede, Netherlands
		}
	}

	% make the title area
	\maketitle

	% No abstract section
	%\begin{abstract}
	%\boldmath
	%\blindtext[1]
	%\end{abstract}
	
	% Note that keywords are not normally used for peerreview papers.
	%\begin{IEEEkeywords}
	%IEEEtran, journal, \LaTeX, paper, template.
	%\end{IEEEkeywords}

	% For peer review papers, you can put extra information on the cover
	% page as needed:
	% \ifCLASSOPTIONpeerreview
	% \begin{center} \bfseries EDICS Category: 3-BBND \end{center}
	% \fi
	%
	% For peerreview papers, this IEEEtran command inserts a page break and
	% creates the second title. It will be ignored for other modes.
	\IEEEpeerreviewmaketitle

	\section{Introduction}
		\indent The era of the Internet of things we are currently living in is still
		improving and everyone owning a means to access Internet can express onself.
		Data published in any way on the network is stored somewhere and can be
		identified. That identification procedure usually includes retrieving more
		information than one can see when posting something. Those pieces of
		information can be anything: IP address, geo-localization, name of the web
		browser and name of the operating system used are few instances. To put it in
		a nutshell, from each data set can be extracted a lot of information which anybody who is provided an access to can do analytics on.\\
		\indent The current context enables to use previously unavailable information
		and create services as to improve users' experience when connected to the
		network. Information about tourists' movements, for instance, could allow
		companies and public institutions to do the necessary so that travelers and
		tourists have a better experience, and even prevent them from doing
		unnecessary expenses. However, useful information cannot be accessed in a
		direct way: it must be extracted. Having at disposal an extracting information
		means which could process huge data sets would be indeed a useful tool in
		order to analyze such data.\\
		\indent For this research topic, we have chosen to output a table with
		relevant attributes. Our table will be filled by analyzing huge data sets of
		tweets which are also used by the \emph{twiqs.nl} website. From each tweet can
		be extracted useful pieces of information, in particular, geolocalization
		details. Some papers propose their method to find users' hometowns like the
		Principal Components Analysis mentionned in paper[1]. As for the technology to
		be used to approach this research topic, Pig Latin, over the Hadoop cluster,
		is the most suited. User defined functions will also be implemented in order
		to increase its efficiency in our favor.
		
	\section{Materials and methods}
	
	\section{Results and discussion}

	% needed in second column of first page if using \IEEEpubid
	%\IEEEpubidadjcol

	\section{Conclusion}

	% Can use something like this to put references on a page
	% by themselves when using endfloat and the captionsoff option.
	\ifCLASSOPTIONcaptionsoff
  		\newpage
	\fi


% trigger a \newpage just before the given reference
% number - used to balance the columns on the last page
% adjust value as needed - may need to be readjusted if
% the document is modified later
%\IEEEtriggeratref{8}
% The "triggered" command can be changed if desired:
%\IEEEtriggercmd{\enlargethispage{-5in}}

% references section

% can use a bibliography generated by BibTeX as a .bbl file
% BibTeX documentation can be easily obtained at:
% http://www.ctan.org/tex-archive/biblio/bibtex/contrib/doc/
% The IEEEtran BibTeX style support page is at:
% http://www.michaelshell.org/tex/ieeetran/bibtex/
%\bibliographystyle{IEEEtran}
% argument is your BibTeX string definitions and bibliography database(s)
%\bibliography{IEEEabrv,../bib/paper}
%
% <OR> manually copy in the resultant .bbl file
% set second argument of \begin to the number of references
% (used to reserve space for the reference number labels box)
\begin{thebibliography}{1} % only the number of digits is important
	\bibitem{1}
		\emph{A Multi-Indicator Approach for Geolocalization of Tweets}, Axel Schulz,
		Aristotelis Hadjakos, Heiko Paulheim, Johannes Nachtwey, and Max Mühlhäuser.
		
	\bibitem{2}
		\emph{Representation and Communication: Challenges in Interpreting Large
		Social Media Datasets}, Mattias Rost, Louise Barkhuus, Henriette Cramer, Barry
		Brown.
		
	\bibitem{3}
		\emph{Photo2Trip: Generating Travel Routes from Geo-Tagged Photos for Trip
		Planning}, Xin Lu, Changhu Wang, Jiang-Ming Yang, Yanwei Pang, Lei Zhang.
		
	\bibitem{4}
		\emph{Knowledge Discovery from Geo-Located Tweets for Supporting Advanced Big
		Data Analytics: A Real-Life Experience}, Alfredo Cuzzocrea, Giuseppe Psaila,
		and Maurizio Toccu.
		
	\bibitem{5}
		\emph{Predicting the Future With Social Media}, Sitaram Asur, Bernado A.
		Huberman.
		
	\bibitem{6}
		\emph{Role of social media in online travel information search}, Zheng Xiang,
		Ulrike Gretzel.
		
	\bibitem{7}
		\emph{Text and Structural Data Mining of Influenza Mentions in Web and Social
		Media}, Courtney D. Corley, Diane J. Cook, Armin R. Mikler and Karan P. Singh.
		
	\bibitem{8}
		\emph{Predicting Tie Strength With Social Media}, Eric Gilbert and Karrie
		Karahalios.		
\end{thebibliography}

% biography section
% 
% If you have an EPS/PDF photo (graphicx package needed) extra braces are
% needed around the contents of the optional argument to biography to prevent
% the LaTeX parser from getting confused when it sees the complicated
% \includegraphics command within an optional argument. (You could create
% your own custom macro containing the \includegraphics command to make things
% simpler here.)
%\begin{biography}[{\includegraphics[width=1in,height=1.25in,clip,keepaspectratio]{mshell}}]{Michael Shell}
% or if you just want to reserve a space for a photo:

\begin{IEEEbiography}[{\includegraphics[width=1in,height=1.25in,clip,keepaspectratio]{picture}}]{John Doe}
\blindtext
\end{IEEEbiography}

% You can push biographies down or up by placing
% a \vfill before or after them. The appropriate
% use of \vfill depends on what kind of text is
% on the last page and whether or not the columns
% are being equalized.

%\vfill

% Can be used to pull up biographies so that the bottom of the last one
% is flush with the other column.
%\enlargethispage{-5in}




% that's all folks
\end{document}


